\documentclass[a4paper,12pt]{exam}
\usepackage[utf8]{inputenc}
\usepackage{amsmath}
\usepackage[portuguese]{babel}
\usepackage{enumerate}
\usepackage{fullpage}
\usepackage{color}

% Add the packages
\usepackage{algorithm}
\usepackage{algpseudocode}
\makeatletter
\renewcommand{\ALG@name}{Algoritmo}
\renewcommand{\listalgorithmname}{List of \ALG@name s}
\makeatother

\begin{document}

\makebox[2\width][l]{Universidade Federal do Ceará - Campus Quixadá}

\makebox[2\width][l]{QXD0010 – Estrutura de Dados -- Turma 03A -- 2021.2}

\makebox[2\width][l]{Prof.~Atílio Gomes}

\bigskip 

\bigskip 

\begin{center}
\fbox{\fbox{\parbox{5.5in}{\centering
\Large Noções de Análise de Algoritmos}}}
\end{center}


\begin{questions}

\question[2.5] Para cada uma das afirmações abaixo, prove se é verdadeiro ou falso,  justificando formalmente (usando definições, manipulações algébricas e implicações se for preciso). \textbf{Atenção:} Está proibido usar limites para resolver esta questão.
\begin{enumerate}[(a)]
\item $10n^2 + 200n + 500/n = O(n^2)$
\item $\lg(100n^3 + 200n+300)^2 = O(\lg n)$
\item $2n^2 - 20n - 50 = \Omega(2n)$ 
\item Seja $C(n,k)$ o número de combinações de $n$ objetos tomados $k$ a $k$. É verdade que $C(n,2) = O(n^2)$? É verdade que $C(n,3) = O(n^3)$?
\end{enumerate}

\question[2.5]
Determine a complexidade de pior caso do algoritmo a seguir:


\begin{algorithm}
\caption{Função F}
\label{FuncF}
\algrenewcommand\algorithmicprocedure{\textbf{Função}}
\algrenewcommand\algorithmicfor{\textbf{para}}
\algrenewcommand\algorithmicdo{\textbf{faça}}
\algrenewcommand\algorithmicwhile{\textbf{enquanto}}
\algrenewcommand\algorithmicend{\textbf{fim}}
\begin{algorithmic}[1]
\Procedure{F}{int L[ ], int n}
    \State $s \gets 0$ 
    \For {$i \gets 0$ \textbf{até} $n-2$}
        \For {$j \gets i+1$ \textbf{até} $n-1$}
            \If {$L[i] > L[j]$}
            \State $s \gets s+1$
            \EndIf
        \EndFor
	\EndFor
	\State \textbf{retorne} $s$
\EndProcedure
\end{algorithmic}
\end{algorithm}


\question[2.5] Faça um algoritmo que verifique se os elementos de um vetor estão ordenados em ordem crescente. Qual a complexidade de pior caso e melhor caso do seu algoritmo? Prove que suas respostas estão corretas. \textbf{Atenção:} Note que eu não estou pedindo para ordenar o vetor.



\question[2.5] A sequencia de Fibonacci é uma sequência de elementos $f_0,f_1,\ldots,f_n$, 
definida do seguinte modo:
\[ f_j =
  \begin{cases}
    j,       & \quad \text{se } 0\leq n\leq 1;\\
    f_{j-1} + f_{j-2},  & \quad \text{se } j > 1.
  \end{cases}
\]
Elaborar um algoritmo iterativo (não recursivo), para determinar o elemento $f_n$ da sequência, cuja complexidade seja linear em $n$ e prove este fato.


\bigskip 

\bigskip 

\bigskip 

\question[1] \textbf{(Bônus)} Considere a seguinte generalização do problema Torre de Hanói. O problema 
agora consiste em $n$ discos de tamanhos distintos e quatro pinos, respectivamente, 
o de origem, o de destino e dois pinos de trabalho (auxiliares). De resto, o problema 
é como no caso de três pinos. Isto é, de início, os discos se encontram todos no 
pino de origem, em ordem decrescente de tamanho, de baixo para cima. O objetivo é empilhar
todos os discos no pino-destino, satisfazendo às condições:
\begin{enumerate}[(i)]
 \item apenas um disco pode ser movido de cada vez;
 \item qualquer disco não pode ser jamais colocado sobre outro de tamanho menor.
\end{enumerate}
Escrever um algoritmo recursivo para resolver essa generalização. O seu programa deve ser escrito em C++ e deve imprimir na tela a sequência de movimentos que resolve esse problema para uma entrada $n$, onde $n$ é o número de discos.


\end{questions}


\end{document}
